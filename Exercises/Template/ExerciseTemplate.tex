\documentclass[11pt]{exam}
\newcommand{\myname}{Ling Song, Hosein Hadipour} %Write your name in here
\newcommand{\myhwtype}{Homework}
\newcommand{\myhwnum}{1} %Homework set number
\newcommand{\myclass}{Cryptanalysis 2021}
\newcommand{\mylecture}{}
\newcommand{\mysection}{}

% Prefix for numedquestion's
\newcommand{\questiontype}{Question}

% Use this if your "written" questions are all under one section
% For example, if the homework handout has Section 5: Written Questions
% and all questions are 5.1, 5.2, 5.3, etc. set this to 5
% Use for 0 no prefix. Redefine as needed per-question.
\newcommand{\writtensection}{0}

\usepackage{amsmath, amsfonts, amsthm, amssymb}  % Some math symbols
\usepackage{enumerate}
\usepackage{enumitem}
\usepackage{graphicx}
\usepackage{hyperref}
\usepackage[all]{xy}
\usepackage{wrapfig}
\usepackage{fancyvrb}
\usepackage[T1]{fontenc}
\usepackage{listings}

\usepackage{centernot}
\usepackage{mathtools}
\DeclarePairedDelimiter{\ceil}{\lceil}{\rceil}
\DeclarePairedDelimiter{\floor}{\lfloor}{\rfloor}
\DeclarePairedDelimiter{\card}{\vert}{\vert}


\setlength{\parindent}{0pt}
\setlength{\parskip}{5pt plus 1pt}
\pagestyle{empty}

\def\indented#1{\list{}{}\item[]}
\let\indented=\endlist

\newcounter{questionCounter}
\newcounter{partCounter}[questionCounter]

\newenvironment{namedquestion}[1][\arabic{questionCounter}]{%
    \addtocounter{questionCounter}{1}%
    \setcounter{partCounter}{0}%
    \vspace{.2in}%
        \noindent{\bf #1}%
    \vspace{0.3em} \hrule \vspace{.1in}%
}{}

\newenvironment{numedquestion}[0]{%
	\stepcounter{questionCounter}%
    \vspace{.2in}%
        \ifx\writtensection\undefined
        \noindent{\bf \questiontype \; \arabic{questionCounter}. }%
        \else
          \if\writtensection0
          \noindent{\bf \questiontype \; \arabic{questionCounter}. }%
          \else
          \noindent{\bf \questiontype \; \writtensection.\arabic{questionCounter} }%
        \fi
    \vspace{0.3em} \hrule \vspace{.1in}%
}{}

\newenvironment{alphaparts}[0]{%
  \begin{enumerate}[label=\textbf{(\alph*)}]
}{\end{enumerate}}

\newenvironment{arabicparts}[0]{%
  \begin{enumerate}[label=\textbf{\arabic{questionCounter}.\arabic*})]
}{\end{enumerate}}

\newenvironment{questionpart}[0]{%
  \item
}{}

\newcommand{\answerbox}[1]{
\begin{framed}
\vspace{#1}
\end{framed}}

\pagestyle{head}

\headrule
\header{\textbf{\myclass\ \mylecture\mysection}}%
{\textbf{\myname}}%
{\textbf{\myhwtype\ \myhwnum}}

\begin{document}
\thispagestyle{plain}
\begin{center}
  {\Large \myclass{} \myhwtype{} \myhwnum} \\
  \myname{}\\
  \today
\end{center}


%Here you can enter answers to homework questions

\begin{numedquestion}

\end{numedquestion}

\begin{numedquestion}
  This question's number will be auto-incremented.
\end{numedquestion}


% if you do not solve some of the questions use this command to increment counter
\setcounter{questionCounter}{4}
\begin{numedquestion}
  Questions 2 and 3 were not solved, this is an answer to question 5.
\end{numedquestion}


% if questions have subparts, use this command
\begin{numedquestion}
  Use the alphaparts environment to for letters.
  \begin{alphaparts}
    \item Part a
    \item Part b
    \item Part c
  \end{alphaparts}
\end{numedquestion}

\begin{numedquestion}
  You can still do things like nesting lists inside of these environments.
  \begin{alphaparts}
    \item part a
      \begin{enumerate}[label=(\roman*)]
        \item making point number 1
        \item making point number 2
        \item making point number 3
      \end{enumerate}
    \item part b
    \item part c
  \end{alphaparts}
\end{numedquestion}

\begin{numedquestion}
  Using the \texttt{description} environment is a great way to typeset induction proofs!
  \begin{description}
    \item[Base Case:]
      Here I have my base case.
    \item[Induction Hypothesis:]
      Assume things to make proof work. 
    \item[Induction Step:]
      Prove all the things.
  \end{description}

  Therefore, we have proven the claim by induction on in the \texttt{description} environment.
\end{numedquestion}



\end{document}